\فصل{مقدمه}

\پاراگراف{}
راه اندازی و استقرار سرویس در صنعت مخابرات به طور سنتی بر این اساس است که اپراتورهای شبکه سخت افزارهای اختصاصی فیزیکی
و تجهیزات لازم برای هر کارکرد در سرویس را در زیرساخت خود مستقر کنند.
فراهم کردن نیازمندی‌های مانند پایداری و کیفیت بالا منجر به اتکای فراهم کنندگان سرویس بر سخت افزارهای اختصاصی می‌شود.
این درحالی است که نیازمندی کاربران به سرویس‌های متنوع و عموما با عمرکوتاه و نرخ بالای ترافیک افزایش یافته است.
بنابراین فراهم کنندگان سرویس‌ها باید مرتبا و به صورت پیوسته تجهیزات فیزیکی جدید را خریده، انبارداری کرده و مستقر کنند.
تمام این عملیات باعث افزایش هزینه‌های فراهم کنندگان سرویس می‌شود[1].
با افزایش تجهیزات، پیدا کردن فضای فیزیکی برای استقرار تجهیزات جدید به مرور دشوارتر می‌شود.
علاوه بر این باید افزایش هزینه و تاخیر ناشی از آموزش کارکنان برای کار با تجهیزات جدید را نیز در نظر گرفت.
بدتر این ‍که هر چه نوآوری سرویس‌ها و فناوری شتاب بیشتری می‍گیرد، چرخه عمر سخت افزارها کوتاه‌تر می‍شود که مانع از ایجاد نوآوری در سرویس‌های شبکه می‌شود[2].

\پاراگراف{}
در روش سنتی استقرار سرویس شبکه، ترافیک کاربر باید از تعدادی کارکرد شبکه به ترتیب معینی عبور کند تا یک مسیر پردازش ترافیک ایجاد شود.
در حال حاضر این کارکردها به صورت سخت افزاری به یکدیگر متصل هستند و ترافیک با استفاده از جداول مسیریابی به سمت آن‌ها هدایت می‌شود.
چالش اصلی این روش در این است که استقرار و تغییر ترتیب کارکردها دشوار است.
به عنوان مثال، به مرور زمان با تغییر شرایط شبکه نیازمند تغییر همبندی و یا مکان کارکردها برای سرویس‌دهی بهتر به کاربران هستیم که نیاز به جا به جایی کارکردها
و تغییر جداول مسیریابی دارد. در روش سنتی این کار سخت و هزینه‌بر است که ممکن است خطاهای بسیاری در آن رخ دهد.
از جنبه دیگر، تغییر سریع سرویس‌های مورد نظر کاربران نیازمند تغییر سریع در ترتیب کارکردها است که در روش فعلی این تغییرات به سختی صورت می‌گیرد.
بنابراین اپراتورهای شبکه نیاز به شبکه‌های قابل برنامه ریزی و ایجاد زنجیره سرویس کارکردها به صورت پویا پیدا کرده‌اند[3]، [4].

\پاراگراف{}
دو فناوری برای پاسخ گویی به این چالش‌ها مطرح شد: مجازی‌سازی کارکرد شبکه\پانویس{Network Function Virtualization}
و زنجیره‌سازی کارکرد سرویس\پانویس{Service Function Chaining}.
مجازی‌سازی کارکرد شبکه با استفاده از مجازی‌سازی کارکردهای شبکه و اجرای آن‌ها بر روی سرورهای استاندارد با توان بالا، امکان اجرای کارکردها
بر روی سخت افزارهای عمومی را فراهم کرده است تا نیاز به تجهیزات سخت افزاری خاص منظوره کاهش یابد.
از طرف دیگر زنجیره‌سازی کارکرد سرویس امکان تعریف زنجیره کارکردها را ارائه می‌کند که ایجاد و انتخاب مسیرهای متفاوت برای پردازش ترافیک
به صورت پویا و بدون ایجاد تغییر در زیرساخت فیزیکی را امکان پذیر می‌کند.
با توجه به این فناوری‌ها، مسائل تحقیقاتی جدیدی مطرح شدند که از مهم ترین آن‌ها می توان تخصیص منابع بهینه به سرویس درخواستی کاربر را نام برد.

\پاراگراف{}
از مهم ترین اهدافی که در حل مسائل تخصیص منابع می‌توان در نظر گرفت، بحث کیفیت سرویس است.
کیفیت سرویس تاثیر مستقیمی بر رضایت کاربر از سرویس‌های یک مرکز داده داشته و از سوی دیگر نقص آن می‌تواند به قرارداد لایه سرویس آسیب زده
و موجب جریمه مرکز داده‌ای شود.
% ارائه دلایل بیشتر می‌تواند به بهبود این پاراگراف کمک کند.

\پاراگراف{}
تحقیقات متعددی در رابطه با تخصیص منابع در معماری مجازی سازی کارکرد شبکه انجام شده است.
تعداد بسیار زیادی از این تحقیقات بحث کیفیت سرویس و یا تاخیر را مدنظر قرار داده‌اند. با این وجود تعداد بسیاری از این تحقیقات فرضیات محدود کننده‌ای
مانند نگاشت تنها یک کارکرد به هر ماشین مجازی، ایجاد حداکثر یک نمونه از هر کارکرد، عدم به اشتراک گذاری کارکردها و \نقاط‌خ
این در حالی است که مراکز داده برای ارائه سرویس بهتر نیاز به استفاده از همه منابع خود داشته و در بسیاری از اوقات نیز نمی‌توان برای ترافیک موردنظر تنها از یک نمونه استفاده کرد.
بنابراین در این رساله فرض شده است که می‌توان از یک کارکرد نمونه‌های مختلف ساخته و از یک ماشین مجازی برای نگاشت بیش از یک کارکرد استفاده نمود.
همچنین در جهت کاهش هزینه‌های مرکز داده‌ای می‌توان از یک نمونه کارکرد برای سرویس‌دهی به چند زنجیره نیز استفاده کرد که در این رساله مدنظر قرار گرفته است.

\پاراگراف{}
در این رساله به تحقق و تمضین توافق‌نامه لایه سرویس، سرویس‌های درخواستی کاربران تمرکز می‌کنیم.
سرویس درخواستی هر کاربر را به صورت مجموعه‌ای از کارکردها که توسط گراف \متن‌لاتین{SFC} با یکدیگر ارتباط دارند در نظر می‌گیریم.
برای استقرار سرویس باید مشخص شود که هر کارکرد باید بر روی چه سرورهایی در شبکه زیرساخت مستقر شود
و پهنای باند لینک‌های زیرساخت چگونه به لینک های بین کارکردها اختصاص یابد.
در این رساله صرفا کارکردهای مجازی را در نظر میگیریم و فرض می‌کنیم که هر کارکرد توسط یک VNF که پیاده سازی نرم‌افزاری آن کارکرد است ارائه می‌شود.
ما فرض می‌کنیم تعدادی درخواست سرویس توسط فراهم‌کننده زیرساخت دریافت شده است. تضمین توافق‌نامه لایه سرویس از سه گام تشکیل شده است: تحقق، تضمین و اثبات.
برای هر یک از درخواست‌ها می‌بایست این مراحل طی شود تا توافق‌نامه لایه سرویس تضمین شود.
% بنابراین از VNF-FG استفاده نکرده و به SFC بسنده می‌کنیم؟ شاید استفاده از VNF-FG برای بحث گراف جذابتر باشد.

\پاراگراف{}
در این رساله ما مساله تحقق و تضمین توافق‌نامه لایه سرویس برای سرویس‌های درخواستی کاربر در مجازی‌سازی کارکردهای شبکه را در نظر گرفته و آن را در قالب سه زیر مساله مرتبط
مورد بررسی قرار می‌دهیم.

\پاراگراف{}
در مساله اول به بحث جایگذاری و تخصیص منابع به سرویس‌های درخواستی در جهت تحقق توافق‌نامه لایه سرویس می‌پردازیم. در این مساله بر خلاف مسائل موجود فرض می‌شود درخواست‌ها
به صورت برخط در اختیار مرکز داده‌ای قرار گرفته و خروجی مساله اول پذیرش یا عدم پذیرش درخواست‌ها می‌باشد.

\پاراگراف{}
همواره در زیرساخت خطاهایی به وجود می‌آید که در نتیجه آن توافق‌نامه لایه سرویس به خطر می‌افتد. در مساله دوم با نظارت بر زیرساخت عملیات‌های لازم پیش و در هنگام
وقوع خطا مشخص می‌شوند تا بتوان توافق‌نامه لایه سرویس را تضمین کرد.

% مساله سوم.
%

\پاراگراف{}
در همه مسائل پیشنهادی نیاز به انتخاب و انجام تعدادی عملیات می‌باشد، بنابراین در حل همه این مسائل از چهارچوب یادگیری تقویتی عمیق\پانویس{Deep Reinforcement Learning}
استفاده می‌شود تا عامل بتواند بهترین عمل را انتخاب و انجام دهد.
با توجه به این موضوع که زیرساخت شبکه به شکل گراف می‌باشد در یادگیری تقویتی عمیق از شبکه‌های عصبی گرافی\پانویس{Graph Neural Network} استفاده می‌شود تا عامل نسبت
به شبکه‌های جدید کارآیی بهتری داشته باشد.

\پاراگراف{}
به صورت خلاصه نوآوری‌های این رساله به شرح زیر می‌باشد:

\شروع{فقرات}
\فقره امکان به اشتراک‌گذاری کارکردها میان چندین زنجیره و در نظر گرفتن پارامتر‌های کیفیت سرویس برای تحقق تفاهم‌نامه لایه سرویس: در جهت مصرف بهینه منابع ممکن است یک
کارکرد میان چندین زنجیره به اشتراک گذاشته شود که نیاز به در نظر گرفتن پارامترهای کیفیت سرویس را دارد چرا که می‌تواند آن‌ها را به مخاطره بیاندازد.
\فقره استفاده از چهارچوب یادگیری تقویتی عمیق بر پایه شبکه‌های عصبی گرافی که می‌تواند کارآیی عامل نسبت به شبکه‌های جدید را افزایش دهد. در تحقیقاتی که به اینجا صورت گرفته است
از الگوریتم‌های یادگیری گرافی استفاده نشده است که در نتیجه آن ویژگی‌های گراف در یادگیری تاثیر نداشته است، در این رساله قصدا داریم با استفاده از الگوریتم‌های یادگیری گرافی
تاثیر ویژگی‌ها گراف شبکه را پررنگ‌تر کنیم.
\فقره در نظر گرفتن بحث‌های نظارتی در مجازی‌سازی کارکرد شبکه برای جلوگیری از خطا به صورت بلادرنگ: خطاهای بسیار در شبکه‌ها رخ می‌دهند که نیاز دارند به آن‌ها رسیدگی شود و
در صورت نیاز حتی از آن‌ها پشیگیری شود. در این رساله این بحث به صورت بلادرنگ در نظر گرفته می‌شود و از سوی دیگر با پیش‌بینی پارامترها از خطاها پیش‌گیری نیز خواهد شد.
\پایان{فقرات}

\پاراگراف{}
در نهایت ساختار رساله به شرحی است که در ادامه می‌آید.
در فصل دوم معماری‌های NFV و SFC و اجزای آن‌ها را شرح می‌دهیم.
در فصل سوم مسائل تحقیقاتی مطرح شده در این معماری‌ها را بررسی می‌کنیم و آن‌ها را از جنبه در نظر گرفتن انرژی با یکدیگر مقایسه می‌کنیم.
در فصل چهارم مسائل پیشنهاد شده در رساله به صورت دقیق شرح داده می‌شوند.
در نهایت در فصل پنجم به روش حل ارائه شده برای حل مسائل می‌پردازیم و فصل ششم به جمع بندی و ارائه زمان بندی انجام رساله اختصاص دارد.
