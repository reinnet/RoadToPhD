\documentclass{article}

\usepackage[localise]{xepersian}

\settextfont{Vazir}

\begin{document}

\قسمت{مساله}

\پاراگراف{}
در این رساله قصد داریم مسائل مرتبط با برآورده ساختن قرارداد لایه سرویس را مورد بحث قرار دهیم. این برآورده ساختن از سه مرحله:

\begin{latin}\begin{enumerate}
   \item Fulfillment
   \item Monitoring
   \item Assurance
\end{enumerate}\end{latin}

تشکیل شده است. یکی از پارامترهای مهم در این برآورده‌سازی بحث کیفیت سرویس می‌باشد.

\پاراگراف{}
تاخیر یکی از پارامترهای مهم در بحث کیفیت سرویس می‌باشد. در شبکه‌های \متن‌لاتین{5G} به این موضوع پرداخته شد اما این پرداخت به اندازه‌ای نبود که بتواند این سرویس‌ها را به صورت عملیاتی پیاده‌سازی کند.
بنابراین یکی از بحث‌های مهم در شبکه‌های \متن‌لاتین{6G} پرداختن به همین بحث تاخیر و سرویس‌های با تاخیر کم می‌باشد.

مساله‌ی اول بحث کیفیت سرویس برای جایگذاری سرویس‌های قطعی در زیر ساخت مجازی‌سازی کارکردهای شبکه است و مساله‌ی دوم بحث بازجایگذاری سرویس‌ها بعد از مانیتور کردن آن‌ها در یک بازه زمانی مشخص است.
در مساله‌ی دوم هدف بهبود جایگذری صورت گرفته در مساله اول خواهد بود.

\پاراگراف{}
در مساله اول که بحث جایگذاری مطرح است نیاز داریم در ابتدا ساختار آنچه که میخواهیم جایگذاری کنیم را مشخص کنیم. این جایگذاری می‌تواند بر پایه \متن‌لاتین{SFC} از استاندارد \متن‌لاتین{IETF} یا \متن‌لاتین{VNF-FG} از \متن‌لاتین{ETSI} باشد.
در مقالاتی مانند این سعی شده است ساختار \متن‌لاتین{SFC}ها به گونه‌ای تغییر کند که توانایی در نظر گرفتن \متن‌لاتین{Load Balancing} و چندین نمونه از یک سرویس را داشته باشد، بحثی مشابه با \متن‌لاتین{Partially and Totally ordered SFC} که پیشتر در ارائه شفاهی دیده بودیم.

در بحث جایگذاری مورد مهم دیگر در رابطه با زیرساخت جایگذاری می‌باشد. زیرساخت می‌تواند برای سرویس‌های مختلف پارامترهای کیفیت سرویس گوناگونی ارائه دهد. به طور مثال می‌توان دو مرکز داده‌ای در نظر گرفت که در کنار جایگذاری نیاز به انتخاب مرکز داده‌ای نیز به وجود می‌آورد.
در اینجا بحث کیفیت سرویس نیز مطرح است. برای بحث کیفیت سرویس نیاز داریم تاخیر سرویس‌ها را در قالب ریاضی فرمول‌بندی کنیم و از این رو بحث \متن‌لاتین{Network Calculus} به ما برای بیان شرایط حدی کمک می‌کند.
مقالات در این حوزه عموما یک رابطه ساده برای منحنی سرویس و منحنی ورود سرویس‌ها، در نظر می‌گیرند و با مفهوم پیچش آن‌ها را به تمام سرویس تعمیم می‌دهند. مقالات کمی در حوزه بهره‌گیری از \متن‌لاتین{Network Calculus} عمیق‌تر از این عمل کرده بودند که عمدتا هم حوزه‌ی آن‌ها بحث‌های \متن‌لاتین{TE} می‌باشد.
در این مرحله می‌توان یک مساله‌ی بهینه‌سازی صحیح مطرح و از روش‌های گوناگون برای حل آن بهره برد. یکی از این روش‌ها استفاده از \متن‌لاتین{Quantum computing} می‌باشد. در عین حال می‌شود از روش‌های یادگیری تقویتی هم بهره جست.

\پاراگراف{}
در مساله دوم قصد داریم سرویس‌های جایگذاری شده را بعد از یک بازه زمانی دوباره جایگذاری کنیم اما بحث اصلی در این رابطه استفاده از پیش‌بینی ترافیک لینک‌ها است.
در واقع متقاضیان سرویس هرگز در رابطه با ترافیک دقیق سیستمشان اطلاعی ندارد. در پیش‌بینی ترافیک دو بحث معیار مطرح است یکی ساختار مکانی و دیگری زمان می‌باشد.
برای حل این مساله نیاز است این دو معیار توامان مدنظر قرار گرفته شوند و از این رو از بحث \متن‌لاتین{Spectral Graphs} و ... استفاده می‌کنیم.
در این مساله نیاز به بحث‌های یادگیری ماشین نیز وجود دارد. در واقع در این مساله سعی خواهیم با استفاده از یک سری زمانی از اطلاعات که از شبکه بدست آمده است حجم ترافیک لینک‌ها را پیش‌بینی کنیم.
در صورتی که برای سرویس‌ها از VNF-FG استفاده کنیم در این مرحله کار سخت‌تری برای پیش‌بینی ترافیک خواهیم داشت.

\قسمت{مقدمه}

\پاراگراف{}
راه اندازی و استقرار سرویس در صنعت مخابرات به طور سنتی بر این اساس است که اپراتورهای شبکه سخت افزارهای اختصاصی فیزیکی
و تجهیزات لازم برای هر کارکرد در سرویس را در زیرساخت خود مستقر کنند.
فراهم کردن نیازمندی‌های مانند پایداری و کیفیت بالا منجر به اتکای فراهم کنندگان سرویس بر سخت افزارهای اختصاصی می‌شود.
این درحالی است که نیازمندی کاربران به سرویس‌های متنوع و عموما با عمرکوتاه و نرخ بالای ترافیک افزایش یافته است.
بنابراین فراهم کنندگان سرویس‌ها باید مرتبا و به صورت پیوسته تجهیزات فیزیکی جدید را خریده، انبارداری کرده و مستقر کنند.
تمام این عملیات باعث افزایش هزینه‌های فراهم کنندگان سرویس می‌شود[1].
با افزایش تجهیزات، پیدا کردن فضای فیزیکی برای استقرار تجهیزات جدید به مرور دشوارتر می‌شود.
علاوه بر این باید افزایش هزینه و تاخیر ناشی از آموزش کارکنان برای کار با تجهیزات جدید را نیز در نظر گرفت.
بدتر این ‍که هر چه نوآوری سرویس‌ها و فناوری شتاب بیشتری می‍گیرد، چرخه عمر سخت افزارها کوتاه‌تر می‍شود که مانع از ایجاد نوآوری در سرویس‌های شبکه می‌شود[2].

\پاراگراف{}
در روش سنتی استقرار سرویس شبکه، ترافیک کاربر باید از تعدادی کارکرد شبکه به ترتیب معینی عبور کند تا یک مسیر پردازش ترافیک ایجاد شود.
در حال حاضر این کارکردها به صورت سخت افزاری به یکدیگر متصل هستند و ترافیک با استفاده از جداول مسیریابی به سمت آن‌ها هدایت می‌شود.
چالش اصلی این روش در این است که استقرار و تغییر ترتیب کارکردها دشوار است.
به عنوان مثال، به مرور زمان با تغییر شرایط شبکه نیازمند تغییر همبندی و یا مکان کارکردها برای سرویس‌دهی بهتر به کاربران هستیم که نیاز به جا به جایی کارکردها
و تغییر جداول مسیریابی دارد. در روش سنتی این کار سخت و هزینه‌بر است که ممکن است خطاهای بسیاری در آن رخ دهد.
از جنبه دیگر، تغییر سریع سرویس‌های مورد نظر کاربران نیازمند تغییر سریع در ترتیب کارکردها است که در روش فعلی این تغییرات به سختی صورت می‌گیرد.
بنابراین اپراتورهای شبکه نیاز به شبکه‌های قابل برنامه ریزی و ایجاد زنجیره سرویس کارکردها به صورت پویا پیدا کرده‌اند[3]، [4].

\پاراگراف{}
دو فناوری برای پاسخ گویی به این چالش‌ها مطرح شد: مجازی‌سازی کارکرد شبکه\پانویس{Network Function Virtualization}
و زنجیره‌سازی کارکرد سرویس\پانویس{Service Function Chaining}.
مجازی‌سازی کارکرد شبکه با استفاده از مجازی‌سازی کارکردهای شبکه و اجرای آن‌ها بر روی سرورهای استاندارد با توان بالا، امکان اجرای کارکردها
بر روی سخت افزارهای عمومی را فراهم کرده است تا نیاز به تجهیزات سخت افزاری خاص منظوره کاهش یابد.
از طرف دیگر زنجیره‌سازی کارکرد سرویس امکان تعریف زنجیره کارکردها را ارائه می‌کند که ایجاد و انتخاب مسیرهای متفاوت برای پردازش ترافیک
به صورت پویا و بدون ایجاد تغییر در زیرساخت فیزیکی را امکان پذیر می‌کند.
با توجه به این فناوری‌ها، مسائل تحقیقاتی جدیدی مطرح شدند که از مهم ترین آن‌ها می توان تخصیص منابع بهینه به سرویس درخواستی کاربر را نام برد.

\end{document}
