\فصل{مرور ادبیات}

\قسمت{مقدمه}

\پاراگراف{}
در این بخش تحقیقات مرتبط با استقرار سرویس و تخصیص منابع در معماری‌های\متن‌لاتین{NFV} و \متن‌لاتین{SFC} را مورد بررسی قرار می‌دهیم.
ابتدا ابعاد مختلف مسائل تحقیقاتی و به خصوص مساله تخصیص منابع را مورد بررسی قرار می‌دهیم.
سپس دسته بندی از تحقیقات انجام شده با توجه به درنظر گرفتن یا نگرفتن تفاهم‌نامه لایه سرویس در تخصیص منابع انجام داده و تحقیقات مرتبط با هر دسته را بررسی خواهیم کرد. در نهایت مهمترین تحقیقات انجام شده غیرمرتبط با مسئله تخصیص منابع را نیز بررسی خواهیم کرد.

\قسمت{ابعاد مختلف مسائل تحقیقاتی}

\پاراگراف{}
در این بخش با توجه به معماری‌های SFC و NFV ابعاد مختلف مسائل تحقیقاتی را مورد بررسی قرار می‌دهیم.
از آنجایی که موضوع این رساله بر مسئله تخصیص منابع تمرکز دارد بر این بخش تمرکز بیشتری خواهیم داشت. در این تحقیق تخصیص منابع را به صورت اختصاص منابع شبکه، پردازشی، محاسباتی و ذخیره سازی به ماشین‌های
مجازی اجراکننده کارکردها و اختصاص پهنای باند به لینک‌های مجازی بین کارکردها بر روی شبکه ارتباطی زیرساخت تعریف می‌کنیم.
در این راستا باید مشخص شود که ماشین‌های مجازی اجرا کننده کارکردها که به عنوان نمونه ایجاد شده از کارکردها شناخته می‌شوند بر روی چه سرورهایی ایجاد شوند.
این فرآیند را نگاشت کارکردها به نمونه‌ها می گوییم. همچنین باید مشخص شود چه میزان پهنای باند از چه لینک هایی به به لینک های مجازی اختصاص یابد.
ممکن است پهنای باند یک لینک، بر روی چند لینک و یا چند مسیر در شبکه زیرساخت اختصاص پیدا کند. به این فرایند نیز نگاشت لینک‌های مجازی گفته می‌شود.
با توجه به توضیحات گفته شده در ادامه ابعاد مختلف مسائل تحقیقاتی را شرح می دهیم.

\زیرقسمت{دیدگاه تعریف مساله}

\پاراگراف{}
معماری آینده اینترنت بر اساس مدل تجاری IaaS1 است که نقش ISP ها به دو نفش فراهم کننده سرویس2 (SP) و فراهم کننده زیرساخت3(IP)، تبدیل می شود[21]. فراهم کننده سرویس مسئول ارائه سرویس انتها به انتها به کاربران بر روی زیرساختی است که از سمت IP ارائه می شود و مسئولیت مدیریت منابع آن را برعهده دارد. بر اساس این دو نقش، مسائل را از دو جنبه می توان دسته بندی کرد:
\شروع{فقرات}
\فقره مسائلی که در آن صرفا بحث سرویس گرفتن از یک یا چند IP مطرح است. از آنجایی که سرویس گیرنده خود تخصیص منابع را انجام نمی دهد، تمرکز این گونه مسئله ها بر روی مسائل قیمت گذاری در یک بازار NFV خواهد بود.
\فقره مسائلی که در آن ها تخصیص بهینه منابع به سرویس نیز مورد توجه است. در این حالت IP از یک مرکز داده متمرکز یا چندین مرکز داده توزیع شده برای ارائه سرویس به کاربران استفاده می کند.  در این بخش وابسته به سطح  انتزاع مسئله، NFVI-PoP را می توان یک سرور یا یک مرکز داده در نظر گرفت. طبیعتا وابسته به سطح انتزاع، همبندی های مختلف شبکه ارتباطی را نیز می توان در نظر گرفت.
\پایان{فقرات}
در هر یک از این دیدگاه ها می توان فرضیاتی را در نظر گرفت که منجر به مسائل متفاوتی می شود. به عنوان مثال در دسته اول نحوه قیمت گذاری، همکاری و یا عدم همکاری IP ها و کاربران را می توان مورد مطالعه قرار داد. در مسائل تخصیص منابع هم اگر یک مرکز داده وجود داشته باشد، درباره شبکه زیرساخت می توان انواع همبندی های Switch centeric و یا Server centric را در نظر گرفت[22] که تاثیر زیادی بر تعریف مسائل دارند. زمانی که چندین مرکز داده توزیع شده از نظر جغرافیایی وجود داشته باشد، مسائل مهمی از جمله نحوه کاهش ترافیک بین مراکز داده مطرح می شود. همانگونه که بیان شد تمرکز این تحقیق بر دسته دوم مسائل یعنی تخصیص بهینه منابع به سرویس است.


\زیرقسمت{انواع مسائل تحقیقاقی}

\پاراگراف{}
همانگونه که پیشتر در بحث مسائل تحقیقاتی در معماری های SFC و NFV اشاره شد به طول کلی سه مسئله اصلی در این معماری ها وجود دارد که ترکیب های مختلف این مسائل با یکدیگر نیز می تواند به عنوان مسائل جدید مطرح گردد:

\شروع{فقرات}
\فقره ساخت زنجیره کارکرد در معماری SFC یا گراف VNF-FG در معماری NFV
\فقره استقرار سرویس و تخصیص منابع بر اساس زنجیره SFC یا گراف VNF-FG
\فقره زمان‌بندی در استفاده از منابع تخصیص یافته به سرویس
\پایان{فقرات}

\پاراگراف{}
مسائلی که بر ساخت زنجیره کارکرد در معماری SFC تمرکز دارند، فرض می‌کنند که کاربر ترتیب صریح زنجیره را ذکر نکرده است و صرفا ترتیب جزئی عبور ترافیک از کارکردها بیان شده است.
به عنوان مثال کاربر بیان می‌کند که زنجیره‌اش شامل سه کارکرد دیواره آتش، NAT و Load Balancer است. ولی ترتیب دقیق آن ها را بیان نکرده و صرفا بیان میکند که Load balancer اخرین کارکرد در زنجیره است.
در این حالت ممکن است NAT قبل یا بعد از دیواره آتش قرار گیرد که وابسته به شرایط شبکه ممکن است حتی در نمونه های ایجاد شده از کارکردها نیز تاثیر گذار باشد.
مشابه این شرایط در ایجاد گراف VNF-FG نیز مطرح می‌شود. از آنجایی که زنجیره‌های متفاوت نحوه استقرار متفاوتی دارند، می توان این مسئله را با استقرار سرویس ترکیب کرد.

\پاراگراف{}
سرویسی که ارتباط کارکردهای آن توسط SFC یا VNF-FG توصیف می‌شود باید در زیرساخت استقرار یابد.
در استقرار سرویس ابتدا باید مشخص شود چه تعداد نمونه از کارکردها بر روی چه سرورهایی باید ایجاد شود و چه میزان پهنای باند از چه لینک‌هایی باید به لینک‌های مجازی اختصاص یابد.
پس از مشخص شدن این موارد منابع به نمونه‌ها و لینک‌های مجازی اختصاص می‌یابد.

\پاراگراف{}
در روند استقرار سرویس، نیازمندی‌هایی مانند نیازمندی‌های کیفیت سرویس اعلام شده نیز باید در نظر گرفته شود.
ممکن است فرض شود که بعضی از منابع، مانند ذخیره‌سازی و یا پهنای‌باند، نامحدود یا از پیش تعیین شده هستند و در روند استقرار سرویس دخالت داده نشوند.
همچنین نمونه‌ها ممکن است ایجاد شوند یا از قبل وجود داشته باشند و صرفا کارکردهای زنجیره به آن ها انتساب پیدا کنند.
انتساب چندین کارکرد یکسان به یک نمونه در حالتی رخ می‌دهد که نمونه ها بتوانند بین کارکردهای زنجیره های مختلف به اشتراک گذاشته شوند و ترافیک زنجیره های مختلف را پردازش کنند.
به عنوان مثال ممکن است فرض کنید دو نمونه برای یک کارکرد دیواره آتش در یک زنجیره ایجاد شده است. زنجیره کارکرد دیگری نیز دارای کارکرد دیواره آتش است و باید برای آن چهار نمونه ایجاد شود.
در این حالت می توان برای آن کارکرد دو نمونه اختصاصی ایجاد کرد و به دو نمونه ایجاد شده فعلی نیز آن را انتساب داد.
در این حالت، آن کارکرد به چهار نمونه انتساب پیدا کرده است و پردازش ترافیک آن توسط چهار نمونه انجام می‌شود.

\پاراگراف{}
در صورتی که در تعداد نمونه‌هایی که می‌توان برای کارکردها ایجاد کرد محدودیتی وجود داشته باشد، ناچار به زمان بندی در استفاده از آن ها خواهیم بود.
در این صورت با درنظر گرفتن ماشین مجازی به عنوان یک منبع، می‌توان زمان‌بندی در استفاده آن را تعریف کرد.
به صورت دقیق تر با اشتراک ماشین‌های مجازی بین چندین کارکرد متفاوت در طول زمان، میتوان سرویس مورد نظر کاربر را برآورده کرد.
این کار به این صورت انجام می‌شود که کارکردهای متفاوت به یک ماشین مجازی انتساب پیدا می کنند و ماشین مجازی در هر زمان صرفا تصویر یکی از کارکردها را اجرا می‌کند.
بنابراین سایر کارکردها باید منتظر بمانند تا اجرای کارکرد فعلی خاتمه یافته، تصویر کارکرد آن ها با کارکرد فعلی تعویض شده و کارکرد آن ها اجرا شود تا بتوانند سرویس خود را دریافت کنند.
تفاوت این حالت با اشتراک نمونه در این است که کارکردهایی که به ماشین مجازی انتساب پیدا کرده اند با یکدیگر متفاوت هستند و ماشین مجازی در هر زمان صرفا یک کارکرد را می‌تواند اجرا کند.
اگر در طول زمان انتساب کارکردها به ماشین مجازی نیز تغییر کند، این مسئله با مسئله تخصیص منابع ترکیب می شود.
در این رساله ما صرفا بر استقرار سرویس و تخصیص منابع تمرکز خواهیم کرد و در آن بحث اشتراک نمونه را نیز مدنظر قرار خواهیم داد.

\قسمت{باید مشخص شود!}

\پاراگراف{}
مساله کیفیت سرویس در جایگذاری یا مهاجرت سرویس‌ها در محل‌هایی با منابع مشترک مانند مراکز داده‌ای بسیار حائر اهمیت است.

\قسمت{مسائل حوزه هوش مصنوعی}

\پاراگراف{}
سال‌ها است که از تکنیک‌های هوش مصنوعی برای انجام عملیات‌های شبکه‌های آینده از مدیریت تا نگهداری و مراقب استفاده می‌شود. از تکنیک‌های یادگیری عمیق
برای مسائلی از جمله جنبه‌های مختلف کنترل ترافیک مانند دسته‌بندی ترافیک شبکه، پیش‌بینی جریان شبکه، پیش‌بینی جابجایی
و شبکه‌های خود سامان‌دهنده\پانویس{Self-Organized Networks} و شبکه‌های رادیویی شناختی\پانویس{Cognitive Radio Network}
استفاده می‌گردد. \مرجع{Benzaid2020}

\پاراگراف{}
با استفاده از تکنیک‌های هوش مصنوعی می‌توان ترافیک ورودی را پیش‌بینی کرده از آن جهت پذیرش سرویس‌ها با رعایت تفاهم‌نامه لایه سرویس استفاده کرد. \مرجع{Benzaid2020}
در \مرجع{Martin2018} محققان با استفاده از یادگیری ماشین و ترکیب آن با \متن‌لاتین{SDN} و \متن‌لاتین{NFV} ترافیک را پیش‌بینی کرده و بر اساس معیارهایی از جمله بار ترافیک،
توپولوژی شبکه و تخصیص منابع را به صورت خودکار برای سرویس‌دهی به تقاضای جدید تغییر می‌دهند. سیستم پیشنهادی \مرجع{Martin2018} از سه قسمت تشکیل شده است، قسمت اول
نقض تفاهم‌نامه لایه سرویس را با یک الگوریتم دسته‌بندی پیش‌بینی می‌کند، قسمت دوم با استفاده از رگراسیون شاخص‌های ترافیکی را پیش‌بینی می‌کند و در نهایت قسمت سوم با استفاده از
داده‌های دو قسمت اول یک مساله بهینه‌سازی را حل می‌کند.

\پاراگراف{}
در \مرجع{Arteaga2018} محققان از یادیگری تقویتی جهت مقیاس‌پذیری و تصمیمات بهتر برای مدیریت تغییرات کارایی استفاده می‌کنند. برای آموزش عامل آن‌ها از \متن‌لاتین{Q Learning}
استفاده می‌کنند. استفاده از الگوریتم‌های یادگیری ماشین به پروژه‌های تحقیقاتی محدود نبوده و پروژه \متن‌لاتین{5G-PPP} قصد دارد یک چهارچوب مدیریت هوشمند برای شبکه‌های نسل پنجم
طراحی کند. در کنار همه مراکز استانداردسازی نیست دست به تعریف استاندارد برای مدیریت خودکار شبکه‌ها زده‌اند که از جمله آن‌ها می‌توان به \متن‌لاتین{ENI} و \متن‌لاتین{ZSM} اشاره کرد.
