% !TeX TS-program = xelatex

\documentclass{beamer}
\usetheme{metropolis}

\usepackage{multicol}

\usepackage{ragged2e} % who justifies the text
\usepackage{xecolor}
\usepackage{amsmath}
%\usefonttheme[onlymath]{serif} %Change the math font

\usepackage{tabularx}
\usepackage{booktabs}
\usepackage[style=numeric,sorting=ynt]{biblatex}
\addbibresource{references.bib}
\usepackage{xepersian}
\settextfont{Vazir}
\setlatintextfont{Roboto}


%---------------------------------------------------------------------------------
% Seetings to force Beamer works with Xepersian and RTL typesetting
%-------------------------------------------------------------------------------
%\raggedleft

% For right to left lists (itemize and enumerate)
\makeatletter
\newcommand{\RTList}{\raggedleft\rightskip\@totalleftmargin}
\makeatother
% Correct the bullet for RTL texts
\setbeamertemplate{itemize item}{\scriptsize\raise1.25pt%
 \hbox{\donotcoloroutermaths$\blacktriangleleft$}} 

% To force beamer use numbering in captions
\setbeamertemplate{caption}[numbered]{}% Number float-like environments

\setbeamertemplate{footline}[frame number]
\setbeamertemplate{section in toc}[circle]
\setbeamertemplate{blocks}[rounded][shadow=true]
\setbeamercolor{block body}{bg=lightgray}
\setbeamercolor{headline}{bg=orange}
\setbeamersize{text margin left=1cm,text margin right=1cm}

\setbeamertemplate{headline}
{
    \begin{beamercolorbox}{section in head/foot}
        \vspace{2pt}\insertnavigation{\paperwidth}\vspace{2pt}
    \end{beamercolorbox}
}
  
%---------------------------------------------------------------------------------
% To force beamer use numbering in captions
\setbeamertemplate{caption}[numbered]{}% Number float-like environments

\setbeamertemplate{footline}
{%
  \leavevmode%
  \hbox{%
    \begin{beamercolorbox}[wd=.333333\paperwidth,ht=2.25ex,dp=1ex,center]{author in head/foot}%
      \usebeamerfont{author in head/foot}\insertshortauthor%
    \end{beamercolorbox}%
    \begin{beamercolorbox}[wd=.333333\paperwidth,ht=2.25ex,dp=1ex,center]{title in head/foot}%
      \usebeamerfont{title in head/foot}\insertshorttitle%
    \end{beamercolorbox}%
    \begin{beamercolorbox}[wd=.333333\paperwidth,ht=2.25ex,dp=1ex,right]{date in head/foot}%
      \usebeamerfont{date in head/foot}\insertsection\hspace*{2em}
      \insertframenumber{} / \inserttotalframenumber\hspace*{2ex}
    \end{beamercolorbox}
  }%
}
\setbeamertemplate{section in toc}[circle]
\setbeamertemplate{blocks}[rounded][shadow=true]
\setbeamercolor{block title}{bg=orange}
\setbeamercolor{block body}{bg=lightgray}
\setbeamersize{text margin left=1cm,text margin right=1cm}

%---------------------------------------------------------------------------------
\title{جایگذاری کارکردهای همراه لبه در بستر کارکاردهای مجازی شبکه}
\subtitle{}
\author{پرهام الوانی}
\institute{%
    دانشکده مهندسی کامپیوتر\\
    دکتر بهادر بخشی
}
\date{\today}
\titlegraphic{\vspace{4.5cm}\flushleft\includegraphics[height=50pt]{images/logo}}

\begin{document}

\makeatletter

\setbeamertemplate{title}{%
  \linespread{1.0}%
  \inserttitle%
  \par%
  \vspace*{0.5em}
}
\setbeamertemplate{subtitle}{%
  \insertsubtitle%
  \par%
  \vspace*{0.5em}
}

\AtBeginSection[]
{%
  \begin{frame}{فهرست}
    \tableofcontents[currentsection]
  \end{frame}
  \begin{frame}
    \begin{center}
      \insertsectionnumber. \insertsection%
    \end{center}
    \usebeamertemplate*{title separator}
  \end{frame}
}

\makeatother


\begin{persian}
%------------------------------------------
% Title frame (0)
%------------------------------------------
{%
  \setbeamertemplate{footline}{}
  \begin{frame}
    \titlepage%
  \end{frame}
}

%-------------------------------------------------------------------------------
\begin{frame}{فهرست}
  \tableofcontents[pausesections]
\end{frame}

%-------------------------------------------------------------------------------
\section{مقدمه}

%-------------------------------------------------------------------------------
\begin{frame}{}
  \begin{itemize}\RTList
    \justifying
    \item اینترنت اشیا اجازه اتصال میلیون شی به یکدیگر و کاربران را می‌دهد.
    \item این داده توسط سرورهای ابری مدیریت می‌شود.
    \item خرابی در لینک بین سرورهای ابری و لایه اینترنت اشیا می‌تواند سرویس را قطع کند.
    \item سرورهای ابری کاربران را بابت پردازش داده‌ها و ... شارژ می‌کنند.
    \item پردازش لبه قصد دارد هزینه‌های پردازش، نگهداری و انتقال داده تا سرورهای ابری را کاهش دهد.
  \end{itemize}
\end{frame}

%-------------------------------------------------------------------------------
\begin{frame}{}
  \begin{itemize}\RTList
    \justifying
    \item اشتراک‌گذاری زیرساخت فیزیکی شبکه
    \item شبکه‌های نرم‌افزار بنیان
    \item مجازی‌سازی کارکردهای شبکه
  \end{itemize}
\end{frame}

%-------------------------------------------------------------------------------
\begin{frame}{آنچه پردازش لبه فراهم می‌آورد}
  \begin{itemize}\RTList
    \item 
  \end{itemize}
\end{frame}

%-------------------------------------------------------------------------------
\section{مراجع}

%-------------------------------------------------------------------------------
\begin{frame}{}
  \printbibliography%
\end{frame}

\end{persian}
\end{document}